\textit{Solución.} Los resultados que se muestran en la \hyperref[fig4]{Figura 4} y la \hyperref[fig5]{Figura 5} son razonables, ya que el modelo linealizado logra aproximar efectivamente el proceso no lineal cuando el punto de operación varía ligeramente. Esto es debido a que la linealización se realiza por medio de series de Taylor, las cuales son conocidas por únicamente ser aproximaciones válidas a funciones matemáticas cuando se evalúa la función cerca del centro de la serie. Esto es claro al comparar las subfiguras (a) y (b) en la \hyperref[fig4]{Figura 4}, ya que en la figura (b) el modelo linealizado se desvía más del modelo real en comparación a la figura (a). siendo la causa principal un cambio más significativo en el punto de operación.


\vspace{1em}
\begin{mdframed}
        Así como el código fuente que genera este reporte y el código de MATLAB para generar las gráficas mostradas anteriormente se encuentran en el siguiente repositorio de Github:
    
    \begin{center}
        \href{https://github.com/Roger-505/tareas-ie0431}{\texttt{https://github.com/Roger-505/tareas-ie0431}}
    \end{center}
\end{mdframed}
