\begin{figure}[!h]
    \centering
    \setlength\extrarowheight{3mm}
    \begin{tabular}{>{\centering\arraybackslash}p{3cm}p{3cm}p{10cm}}
        \toprule\\[-2.5em]
        Abreviación & Clasificación & Descripción\\
        \midrule
        $K_r$ & Ganancia & Convierte el valor de referencia de un nivel de pH a una corriente a partir de una escala conocida\\
        $C(s)$ & Controlador & En base al error, controla el pH del líquido en el tanque por medio de una señal de control\\
        $G_B(s)$ & Actuador/ \newline Elemento final de control & En base a la señal de control, modifica el porcentaje de apertura de la válvula de control de pH para modificar el caudal de la sustancia B\\
        $G_P(s)$ & Proceso & Tanque con solución acuosa, cuyo pH es modificado en base al caudal de sustancia ácida/alkalina que le ingrese\\
        $G_S(s)$ & Sensor & pHímetro, que convierte el nivel de pH a una señal de tensión DC\\
        $G_t(s)$ & Transmisor & Convierte una señal de tensión DC correspondiente al nivel de pH, a una señal de corriente\\
        $G_T(s)$ & Perturbación & Asociada a la perturbación $d_T(s)$\\
        $G_{dp}(s)$ & Perturbación & Asociada a la perturbación $d_{dp}(s)$\\
        \bottomrule
    \end{tabular}
    \captionof{table}{Funciones de transferencia del sistema de control realimentado}
    \label{t1}
\end{figure}

